\documentclass[../thesis.tex]{subfiles}

\begin{document}

\chapter{Results}\label{results}
This chapter presents the results of the study
which will be used to evaluate
the prototype with respect to whether or not it makes
vocal programming more efficient.


\section{Interview (Part 1)}%
\label{sec:interview_1}

\subsection{Prior Experience}%
\label{sub:prior_experience}
Participants were asked about their level of experience with programming in general, Talon, 
as well as the tools that were the targets for the prototype (Elm and Vim),
see Table~\ref{tab:experience}.

It should be noted that the amount of time they had been using Talon
does not necessarily correspond with their level of proficiency.
How fast users become proficient with Talon depends on many factors
such as employment status and how much their health allows them to invest time into learning it.
P4 have been using Talon for the shortest amount of time (3 month), but was among the most proficient
out of the six participants.
P3 was only one that was using Dragon before it started using Talon.

None of the participants had an experience with Elm, although two knew Haskell.
This did not have any effect on the study beyond one participant
being momentarily confused by the result of the \textit{call} command
as they did not know the syntax for function invocation in Elm.

The group had mixed experience levels with Vim.
The less experienced participants would make a few
more mistakes during the test due to forgetting to enter insert mode, 
but it did not impact the study in any significant way.

Five of the six participants had already been programming for many years
and were very familiar with the concepts of parse trees and nodes
in the context of programming languages.
P2 who reported the lowest amount of programming experience
was at least somewhat familiar with the concepts and had no problems
understanding the ideas behind the system.

\begin{table}[htpb]
    \centering
    \label{tab:experience}
    \begin{tabular}{|c|c|c|c|c}
           & Programming&Talon&Elm&Vim\\
        P1 & 16 years&1 year&Knows Haskell&No Experience\\
        P2 & 1-2 years&1-2 years&No Experience&Little Experience\\
        P3 & 14 years&1 year*&No Experience&No Experience\\
        P4 & 25 years&3 months&No Experience&Very Experienced\\
        P5 & 20 years&1 year&Knows Haskell&Somewhat Experienced\\
        P6 & 25 years&1 year&No Experience&Very Experienced\\
    \end{tabular}
    \caption{Participants Experience}
\end{table}


\begin{table}[htpb]
    \centering
    \label{tab:setup}
    \begin{tabular}{|c|c|c|c|c}
        & Talon Usage&Speech Engine&Command Set&Platform\\
        P1 &80\%&Conformer&knausj&Linux\\
        P2 &80\%*&Dragon&knausj&Mac\\
        P3 &90\%*&Dragon&knausj&Windows\\
        P4 &60\%&Conformer&knausj*&Mac\\
        P5 &50\%&Conformer&knausj*&Mac\\
        P6 &99\%&Conformer&knausj&Linux\\
    \end{tabular}
    \caption{Participants Setup}
\end{table}

\subsection{Issues With Using Talon}%
\label{sub:issues_with_talon}
Before having them try the prototype, participants were asked about what they perceived to be the most significant challenge
with using Talon currently.
The issues mentioned are visualized in Figure~\ref{fig:challenges_cloud}.

The most common issue that is directly related to the task of programming was navigating within a single file.
Participants expressed some frustration with the way they were navigating
code which often involved using line numbers, and iteratively stepping through the words in the line.
This ties into the issue mentioned by P1 of voice control having a significantly slower feedback loop
compared to keyboard and mouse.
Navigating by voice the same way you would with keyboard is very repetitive
and causes more vocal strain, which were major concerns for P2 and P6.
Navigating in this fashion also increases the cognitive load as they must execute more
commands in order to achieve the task.

All participants stated that they are overall slower with voice control 
than they were with keyboard, although they find Talon faster for certain tasks 
that they do often and are familiar with.
The major time loss seems to be misrecognition of dictated text due to
homophones and identifiers often including domain specific words and abbreviations.
Identifiers being misrecognized forces the user to resort to spelling, 
which is repetitive and make the voice control feel low-level and verbose.

\begin{figure}[htpb]
    \centering
    \includegraphics[width=0.8\linewidth]{images/word_cloud.png}
    \caption{Word Cloud of Challenges}%
    \label{fig:challenges_cloud}
\end{figure}
\end{document}

\section{Usability Test}%
\label{sec:usability_test}



\section{Interviews (Part 2)}%
\label{sec:interviews_part_2_}

\subsection{General Impressions}%
\label{sub:general_impressions}
All participants reported a positive impression of the prototype.
Several immediately stated that it seems really useful and that they were
very impressed.
P3 describes the system as ``accurate and intuitive'',
and P1 used the phrase ``game changer''.

\subsection{Was the Prototype Intuitive?}%
\label{sub:was_the_prototype_intuitive_}
All participants reported that the prototype expected as they expected
and was overall intuitive.
P1, P3 and P5 did however point out an inconsistency in the Talon grammar
used during the test.
As they noted themselves, this is not really an issue because
the grammar would be part of the user configuration which they would be
able to easily change or rewrite completely themselves.

\subsection{Favorite Features}%
\label{sub:favorite_features}
Participants were asked if they found any future particularly useful.
Five out of six participants mentioned navigating to nodes by their
associated identifier as a standout feature.
P3 was most impressed by the symbol awareness, stating:
``It was very useful that formatting [of identifiers] gets picked up automatically such as with onClick''.
The example they are referring to is the \textit{call {user.functions}} command
which allow them to say ``call on click'' to produce ``onClick''<space>.
P2 also mentioned selecting nodes easily would be very useful
for the kinds of tasks they do often.
P5 mention both structural navigation and editing, but also stated that
``exposing identifiers is the most important feature'' arguing that
this is the part of the system that is the most reusable
and that other people can build upon it in their own scripts.
This sentiment was also expressed by P1.
P6 said that ``everything was better than what I'm currently using''.

\subsection{Requested Features}%
\label{sub:requested_features}
The most requested feature was, as expected, support for more languages and editors
as well as adding awareness of more kinds of syntax nodes.
The fact that the prototype is only aware of symbols within a single file
and cannot perform project wide navigation is a clear limitation, as it was pointed out by several participants.

Being able to move nodes in a single command was mentioned by P2 and P4.
This is a feature that can be implemented user level
and does not require any extensions to the backend.

P6 requested the ability to dictate identifiers containing abbreviations, which was discussed in
Section~\ref{overrides}.

\subsection{System Relevance}%
\label{sub:system_relevance}
In order to evaluate the systems effectiveness in making vocal programming more efficient,
participants were asked if they feel like the system addresses some of the challenges you have been experiencing with vocal programming.
Four out of six participants immediately answered yes.
P1 reiterates how much of an improvement the navigation is,
and P5 says it makes vocal programming feel more high-level.

P4 is a bit hesitant to say whether or not it directly addresses existing challenges
, but also says that it is still an improvement.
This participant does note that they already have a very advanced solution for navigation
in their editor through a plug-in that have not yet been released.
Some of the features of this system therefore overlap with their solution, but
my solution might be better for global navigation.

P2 answered that for their particular use case, there issues was not
addressed as directly, but they still think it's a useful addition.
The navigational features were again mentioned as being particularly useful.

\subsection{System Interest}%
\label{sub:system_interest}
When asked if they would be interested in using a finished version of the system,
all participants said yes.
Several also noted that the system would not need to be finished,
and they would use it as soon as it supports their languages
and editors.
P2, P4 and P5 expressed interest in contributing to the project
if I was interested in working on it after the thesis is finished.
P3 stated they would like to see the finished thesis.



