\documentclass[../thesis.tex]{subfiles}

\begin{document}

\chapter{Methodology and Evaluation}\label{methodology_and_evaluation}
This chapter outlines the research methods used in this project,
and how the study was designed and executed in order to evaluate effectiveness of the prototype.
The results of the study will then be presented and analyzed, followed by
discussion of alternative methods that could have been used.

% I might use a mix of quantitative and qualitative methods. (multi- strategy) % read creswell,2018
% I could hold interviews with users of Talon to gather both types of data.

% \paragraph{Speed Testing}
% I can test the users programming speed with and without my system.
% Here I must consider factors such as performance anxiety.
% The programs they would be asked to dictate would have to be carefully crafted in order to have a consistent difficulty level
% across the test. 
% Should they dictate the same program twice, or two different? How can I be sure they are similarly difficult?
% Should I change the order of the two tests?
% This might well be a good measurement. I'll discuss why this paragraph.


\section{Research Methods}
Quantitative research is an approach for ``testing objective theories by examining the relations between variables'', whereas qualitative research is concerned with ``exploring and understanding the meaning of individuals or groups in the context of a social or human problem''. \parencite{creswell2018}

When choosing a research approach it is important to keep in mind the goal of the survey.
The overall goal of the project is to improve the effectiveness of vocal programming by the use of modern programming language tooling.
The prototype should serve as proof of the effectiveness of the tools chosen to achieve this task.
In order to validate that this is in fact achieved, the system will have to be tested with actual users.
As will be discussed in the section on~\nameref{alternative_methods}, there are quantitative measures that could be used which would make a ``Multi-strategy'' approach suitable.
The challenges with these however lead me to implement a more qualitative approach.


% Explain what qualitative research is
% Explain what quantitative researchers
% Explain why I chose qualitative, refer to alt methods section

\subsection{Interviews}

\subsection{Usability Testing}

\section{Participants and Recruitment}

\section{The Study}


% "focus on discovering and understanding the experiences, perspectives, and
% thoughts of participants - that is, qualitative research explores the meaning, purpose
% or reality" (Harwell, 2011).

% I can gather qualitative data by having the user program using my system and ask them whether or not they find using it to be an improvement.
% The disadvantage to this approach is that the system might have a learning curve which makes the initial impression
% worse than it would have been over time.

\section{Alternative Methods}\label{alternative_methods}
This section presents a few alternative methods that could have been used for this project.
\begin{itemize}
    \item User test, coding speed
    \item analyze identifier cost
\end{itemize}
% One quantitative measure I could use is to analyse larger codebases to see how the predicted time to speak common identifiers change.
% If I can count the number of syllables in a word I can compare the length of normal phrases need to produce a given function in a code base and compare
% it to that of my system. This analysis can be weighted by the frequency of said identifiers.
% The advantage of this approach is simplicity in that I don't depend on external users.
% One interesting point here is to see how the result of this analysis relate to the results gathered from interviews.
% \subsection{Data Collection}
% Analyze the Elm implementation of real world app (https://github.com/rtfeldman/elm-spa-example)

\end{document}
