\documentclass[../thesis.tex]{subfiles}

\chapter{Discussion}%
\label{cha:discussion}
This chapter discusses the results presented in Chapter~\ref{results}.
First, the results will be discussed and used to answer the research question.
Then there will be a discussion have implications for theory and practice, followed by
a section on the limitations of this study.

\section{Increase in programming efficiency}%
\label{sec:increase_in_programming_efficiency}
The response to the prototype was overwhelmingly positive.


\section{Structural editing by voice}%
\label{sec:structural_editing_by_voice}



\section{Implications for Theory}%
\label{sec:implications_for_theory}
The capabilities of TreeSitter clearly extends far beyond the original
intended use case of syntax highlighting in code editors.
The very same core features that make it ideal for that use case
also make it an excellent choice for any other system
that deal with syntactic analysis of frequently changing programs.

Talon is an extremely flexible, general purpose voice control system.
The results of this study shows that it is easy to implement
fine-tuned domain specific voice control on top of Talon, as opposed to
making a specialized voice recognition system from scratch.

\section{Implications for Practice}%
\label{sec:implication_for_practice}
The prototype developed in this study is already very usable
for people using Vim and Elm.
I have shown that adding more languages should be very simple,
and the size of the Vim extension suggests that adding more editors
should also be relatively simple.

With a little extra effort, the work presented in this thesis
can address several issues that users of Talon are currently facing.



\section{Limitations}%
\label{sec:limitations}

% Longevity
% Numbers of participants
% did not use lsp
% stress test
% hole oriented editing
