\documentclass[a4paper,english]{ifimaster}

\usepackage[utf8]{inputenc}
\usepackage{babel,duomasterforside}
\usepackage{hyperref}

\usepackage[backend=biber,style=authoryear]{biblatex}
\usepackage{minted}

\nonstopmode{}
\addbibresource{citations.bib}

\title{Master Thesis}
\subtitle{Leveraging Language Tooling for Better Voice Coding}
\author{Gaute Berge}

\begin{document}

\duoforside[dept={Department of Informatics},
program={Informatics: Programming and System Architecture},
long]

\frontmatter{}
\chapter*{Abstract}

\tableofcontents{}
\listoffigures{}
\listoftables{}

% \chapter*{Preface}

\mainmatter{}
\chapter{Background and Motivation}

In this section I will cover the history of voice coding (Dragon, NatLink, Dragonfly, Caster etc\ldots)
as well as the state of the art language tooling. (LSP and TreeSitter).
I will also cover something about the Elm programming language, as I will be using that in my analysis.
% Donald Knuth often says smart stuff ~\parencite{Knuth:2007:CPA:1283920.1283929}.



\section{Introduction}

\section{Injuries}
Here I will cover research related to injuries such as RSI in the tech industry.

\section{Dragon NaturallySpeaking}
The worlds first large-vocabulary general purpose dictation system.
Dragon is in many ways the foundation for many of tools like Talon.
Some voice systems use it as just a speach engine, but it also provides other functionality like mouse control.
Available for mac and windows, but discontinued for mac.% when??
The mac version is still available on eBay, and still works. % maybe something here about the download stuff.

\section{NatLink}
While dragon alone is a very powerful dictation system, it's customization ability is limited.
Dragon does provide a simple scripting language in the pro edition, but it has very limited capabilities compared to
moden general purpose scripting languages.
In 1999, NatLink was developed to solve this problem. It is a macro system for dragon which lets the user write macros that can be triggered by voice commands
in python.~\parencite{gould2001implementation}
Later on, more high level apis were built on top os natlink such as dragonfly and caster. % Not really precise. Need to look into history here.

\section{Command First vs Data First}
Two different dication strategies.
Command first = Output only triggered when a command is triggered by speaking a particular phrase.
Data first = The user can speak without keywords. The system should infer capitalization and punctuation.
Here i will describe the tradeoff between these two.

\section{VoiceCode}
Seems like this was a popular dictaion system at some point. There are papers on it, so might be worth dicussing.


\section{TalonVoice}
Here I cover everything I can about how to use Talon and how the project is organized.
I wrote some in the essay, but this should be updated for the new API.

\subsection{Current state of the project}
\subsection{Distribution and Monetization}
\subsection{Community}
\subsection{Learning Resources and Documentation}

\section{Writing Voice Commands with Talon}
\subsection{Rules}
\subsection{Actions}
\subsection{Contexts}
\subsection{Dynamic Lists}

\section{Using Talon}
In this section I will cover my workflow with Talon. This is relevant because the reader needs to see the advantages and disadvantages of using talon compared to using a keboard.
\subsection{Text Editing}
Here I will cover how I use Talon can be integrated with Vim and IntelliJ, and also cover how to edit text without editor integration.
\subsection{Web Browsing}
Vimium and Surfingkeys. How are these plugins used.
\subsection{Using The Terminal}

\section{PL Theory}
As much of the work is related to analysing programs, I should cover some general stuff abuut languages.
Not sure how much detail is appropriate here, but could be a source for a lot of content.
Possible subjects: Parsing, type checking/inference, scopes, etc.

\section{Language Server Protcol}
Client-server based approach to reuable language IDE features.
I'll compare this to the traditional IDE approach (Eclipse / Intellij)
and discuss how this can be used by other tools than editors.

\section{TreeSitter}
TreeSitter is a system for incremental parsing developed by GitHub.
It was conceved as a solution to the problem of instant semantic syntax highlighting,
but is also used in the implementation of some language servers.

\section{Elm}
In this thisis I will be using Elm as the subject of the alysis.
Elm is a feature rich, but relatively simple language.
Some knowledge of Elm is useful for the reader, no experience is required.
Primaraly I will cover the features of Elm that introduces new identefiers into the scope,
as these will be relevant when generating voice commands later.

\chapter{The project}

\section{Methodology And Evaluation}

\backmatter{}
\printbibliography{}

\end{document}
