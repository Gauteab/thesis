\documentclass[a4paper,english]{ifimaster}

\usepackage[utf8]{inputenc}
\usepackage{babel,duomasterforside}
\usepackage{hyperref}

\usepackage[backend=biber,style=authoryear]{biblatex}
\usepackage{minted}

\nonstopmode
\addbibresource{citations.bib}

\title{Master Thesis}
\subtitle{Structural Code Editing With Assistive Technologies}
\author{Gaute Berge}

\begin{document}

\duoforside[dept={Department of Informatics},
program={Informatics: Programming and System Architecture},
long]

\frontmatter{}
% \chapter*{Abstract}

\tableofcontents{}
% \listoffigures{}
% \listoftables{}

% \chapter*{Preface}

\mainmatter{}
\chapter{Background and Motivation}
% Donald Knuth often says smart stuff ~\parencite{Knuth:2007:CPA:1283920.1283929}.

\section{Introduction}
Software developers spend a very large portion of their day in front of a computer.
While this is necessary for them to do their jobs, it also puts them at higher risk of acquiring certain injuries.
Many of the actions associated with using computer are also associated with risk of injury.
Symptoms of these injuries commonly affect wrists, shoulders and the neck, but can also extend to other parts of the body.
Prolonged use of the mouse can famously lead to carpal tunnel syndrome, commonly known as "mouse arm".
Injuries caused by prolonged static stress are collectively known as Repetitive Strain Injuries, or RSIs for short.

% injured programmers might lose their jobs
When an RSI is first acquired, for many people, there are limited alternatives to taking a leave of absence.
The recovery period ranges from a few months to a few years, and because of the nature of the injury
it is very difficult to keep working while recovering.

% there is a solution - Talon
Lately, however, there have been an emergence of solutions that allow for the control of computers
with alternative input methods such as voice and eye control.
Advancements in Machine Learning over the past decade have made speech recognition extremely accurate,
allowing for both accurate prose dictation and voice command recognition.
This has enabled the popularization of digital assistants such as Google Assistant and Amazon Alexa.
While these systems work well, they are very specialized to answer certain queries and controlling smart devices.
There are however other much less well-known systems that are designed to solve the more general problem
of controlling a computer without any help from keyboard or mouse.
One such system is Talon, which i will be using throughout this work.

% this is new, not researched

% different way of editing code - (no mouse)
Perhaps not so surprisingly, changing the fundamental input method also changes the requirements and preferences of the user.
Some actions such as holding the arrow key until the cursor is at the desired position does not translate well to voice input.
For this reason, many users prefer to use nonstandard editing schemes, such as that of Emacs and Vim.
These editors have a more command-based approach to navigating files, which are less reliant on the mouse
and more efficient for users who prefer to primarily use the keyboard.

% structural editing might be useful
Structural code editors are a type of editor that is radically different from the textbased editors that most programmers use.
These editors provide interfaces that allows the user to more directly interact with the underlying tree structure of the program,
rather than the textual representation.
By interacting with the tree directly, some action such as selecting whole expressions can be simplified.
This also make it possible to ensure that the user cannot create syntactically invalid programs.
Because of the different advantages and disadvantages of using alternative input methods strengths of such editors 
might be amplified.

% structural editors are not designed for use without mouse
Structural editors are not usually approach with coding speed as the main design goal.
Often the interface heavily relies on using the mouse, which can make them hard or impossible to use.
In this research I will investigate the needs and requirements of users of alternative input methods
and explore the possibility of using structural editing to improve the efficiency of coding.
More generally the goal of this project is to discover patterns in software engineering and design
that will improve usability for programmers with disabilities.
% i will try to find out structural editing can be useful for disabled programmers.

\section{TalonVoice}
Talon is a system that enables people to use a computer with alternative, hands-free input methods.
It supports controls using voice commands, noise input, and eye tracking.
The features of Talon are customizable and extendable through a python scripting api.
Through this api users can define custom actions that responds to certain noises, voice commands, or where the user is looking.
These actions can be very simple, such as emulating a sequence of keystrokes, but can also be arbitrary python functions
that can send commands to the operating system, or make network requests.
This section should serve as an overview of Talon and how it is used to make different features of a computer accessible to people with disabilities.

\subsection{Current state of the project}
There are two versions of Talon available: a public release and a private beta.
The public release (version 0.0.8.42) is only fully supported on macOS versions 10.11 through 10.14.
This distribution also works on newer versions of macOS, but you cannot use the built-in speech recognition engine.
In the private beta there is support for both newer version of mac os, as well as Windows and Linux.
This version also comes with improvements to the built-in speech recognition engine, and a rework of the current scripting api, including a domain specific scripting language that can be used in conjunction with python.


\subsection{Distribution and Monetization}
Talon is freely distributed, but closed source.
Its creator, Ryan Hileman, works full-time on the project and is solely funded through Patreon which is a site that allows people to make monthly donations.
There are currently 225 users donating on Patreon, pledging a combined \$3,526 per month[24.04.20].

\subsection{Community}
The community is organized through a Slack channel where users can discuss topics related to using Talon such as equipment, health, etc.
This is also where users go to seek help and discuss issues with the project.
Hileman is usually available to answer question and help users.

\subsection{Learning Resources and Documentation}
New users will face a quite steep learning curve.
The official documentation is largely incomplete, so users are encouraged to learn the api primarily through looking at examples.
Most new users we learn the basics off using Talon by reading Emily Shea's guide "Getting Started with Voice Driven Development".
This guide covers the most essential concepts you need to learn to use Talon such as inputting basic keyboard combinations, dictating with formatters, bringing up the command reference, as well as some general tips for becoming proficient with Talon.
The official documentation also refers to the official examples repository which provides a handful of examples that users can download directly, or use as reference for writing their own scripts.
A common way to get started to download a larger command set such as talon\_community which comes with a lot of functionality out-of-the-box.
There is also an unofficial documentation repository.
While this covers more than the official documentation, and is very useful, it is also incomplete, and is not guaranteed to be immediately updated if the api is changed.

\section{Writing Voice Commands with Talon}
Talon does not come with any voice commands built-in.
In the directory where Talon is installed, there is a folder called \textbf{user}.
When scripts are added to this folder, they are immediately loaded, and the system we listen for any voice commands defined in the scripts.
Because of this I find it appropriate to start with how voice commands are defined before explaining how they are used in practice.
Here is a basic example of how to define a voice command set:

\begin{minted}{python}
from talon.voice import Context
context = Context("example")
context.keymap({
	# say "is equal", output is " == "
	"is equal": " == ",

	# press keys using Key
	"select all": Key("cmd-a")
})
\end{minted}
The above code defines a context with the name "example", and attaches is a keymap.
The keymap is where voice commands are defined.
It is a dictionary that maps rules, represented as strings, to actions which can be strings, Keys, functions, or lists of actions.
In the rest of this section I will explain how voice commands are defined in the keymap
and organized into contexts.

\subsection{Rules}
Rules describe sets of phrases that will trigger a certain action.
They can be simple phrases such as the ones seen above, in which case the action is triggered when that exact phrase is recognized.
It is also possible to define complex rules using a syntax very similar to regular expressions with a few extensions for
capturing dictation input and selecting words from a predefined list.
The following table shows be available syntactic constructs that are used to make rules.
\begin{center}
\begin{tabular}{ c | c }
    () & grouping   \\
    $[]$ & optional   \\
    $\mid$ & alternation  \\
    $*$ & zero or more   \\
    $+$ & one or more     \\
    $<dgndictation>$&arbitrary phrase \\
    $<dgndictation>++$&arbitrary phrase (greedy) \\
    $<dgnwords>$&arbitrary word \\
\end{tabular}
\end{center}
Here are some examples of usage along with an explanation:
\begin{minted}{python}
context = Context("example")
context.keymap({
    # Recognizes "func" and "function"
    "(func | function)": ...

    # Recognizes "class" and "new class"
    "[new] class": ...

    # Recognizes "good", "very good", "very very good", etc...
    "very* good": ...

    # Recognizes "very good", "very very good", etc... (not "good")
    "very+ good": ...

    # Recognizes "say <FREE DICTATION>"
    "say <dgndictation>": ...
})
\end{minted}
dgndictation lets the user capture normal dictation and use that as input to the action.
It's variants dgndictation++ and dgnwords does the same but vary in how other rules are triggered
during dictation.

\subsection{Actions}
Actions describe what will happen once the corresponding rule has been recognized.
They can be described using one of the following python constructs:
\begin{center}
\begin{tabular}{ c | c }
    string & insert text   \\
    Key & press keys   \\
    function & function invocation  \\
    list & sequence of actions   
\end{tabular}
\end{center}
Inserting text corresponds to pressing the keys in sequence.
Key objects allows you to press any key with modifiers.
When a function is provided as an action the function will be invoked with an object
containing information about the captured dictation as an argument.
The function can then do processing on the dictation input and use that to carry out the desired action.
List are used to to perform multiple actions in sequence and are often used when programming macros with Talon.

\subsection{Contexts}
Voice comman organized into contexts.
When defining a context you can specify a predicate function that is used to 
decide whnr not the context should be active based on the currently active
application or window.
This is a very important feature for two major reasons.
Firstly, it makes sure that commands intended for a certain context is not
accidentally triggered in another.
Commands defined in an inactive context will not be recognized by the speech engine.
Secondly, it allows for commands to be overloaded between different contexts.
For example, when programming in different languages there are actions
that are conceptually identical, but are syntactically different.
One might have a command \textbf{function} that will insert the keyword associated with
defining a function for the given programming language, followed by a space.
Such a command can be defined as such:
\begin{minted}{python}
# javascript.py
context = Context("javascript", func=is_filetype(("js", )))
context.keymap({
    "function": "function "
})
\end{minted}
\begin{minted}{python}
# python.py
context = Context("python", func=is_filetype(("py", )))
context.keymap({
    "function": "def "
})
\end{minted}
The function \textbf{is\_filetype} is defined in the \textbf{utils} module provided by \textbf{talon\_community}.




% \section{Using Talon}
% \subsection{Text Editing}
% \subsection{Web Browsing}
% \subsection{Using The Terminal}


\section{Mouseless Code Editing}
Most programs where the user is to interact with text provides a similar editing scheme.
The text is laid out on screen with a thin cursor that indicates where the next character you type will be inserted.
The mouse can be used for navigating, selecting and reordering text.
Selecting single words, lines, and larger blocks of text can be done with double-click, triple-click, and holding and dragging the cursor respectively.
When a block of text is selected it can be moved by dragging it with the mouse.
Most of these actions can also be performed using just the keyboard.
By using a combination of keyboard modifiers and arrow/navigation keys, the user can jump, select and delete whole text objects such as words, lines and paragraphs.
This editing scheme has the advantage of being intuitive and easy to understand.
For many use cases however, this editing scheme leaves a lot to be desired in terms of efficiency.
This is especially true for code editing.  % \textbf{Explain why this is the case?}
In the following section I will discuss a radically different editing scheme and how this affects users of assistive technologies.

\subsection{Vim}
Vim is an originaly terminal based text editor available on most platforms.
It is the successor of the "vi" editor and the name stands for "vi-improved".
Vim is a modal editor which means that the editing scheme is centered around different modes.
Vim supports twelve different modes, six of which are considered main modes, while the other six
are variations of the main modes.
The two most important modes are Normal and Insert.
Insert mode works like a standard text editor.
When in normal mode, which is the mode you are in when opening the editor, keystrokes will not enter text, but rather execute commands.
In normal mode the keys h,j,k,l will move the cursor left, down, up and right respectively.
Vim is aware of text objects such as words, lines and paragraphs and supports many commands for moving between and operating on these objects.
\textbf{w} moves the cursor to the start of the next word, \textbf{e} moves to this end of the current word et cetera.
By composing verbs (such as \textbf{d} for delete) with movements and descriptions of objects you gain access to a large
set of commands to make complex transformations on text.
For example \textbf{dw} will delete a word, \textbf{dip} (delete inner paragraph) will delete a paragraph and so on.
With the huge amount of much more complicated commands available Vim allows for extremely efficient and ergonomic text editing.
This editing scheme has become very influential and is integrated in other kinds of applications such as browser plug-ins
that allows for mouse-less browsing.



\section{Structural Editing}
Structural editing is a technique used by some code editors to allow users to directly manipulate the underlying structure of the program.
Computer programs typically have a textual representation that is reasonable both by humans and compilers.
When a compiler reads a program the program text is converted to the district are known as the Abstract Syntax Tree (AST), during the parsing phase.
The AST contains all the information about the program necessary to produce the executable output program.
When modifying a program the desired outcome is always the correct change in the underlying AST, however it is not always
trivial to map of single text transformation to a AST transformation.
One of the goals of structural editing is to reduce the amount of invalid states the program needs to go through
before the desired modification is completed.
For this reason it is a popular technique in editors designed specifically for beginners.
One very well-known example is "Scratch" which is an educational tool for teaching children to code.
It provides a drag-and-drop interface enriched with visual cues that makes it easy to
put together structurally valid programs.
Scratch is an example of a purely structure oriented code editor.
It does not provide enter text based interface to the language.
There are also hybrid approaches that adds a structural layer on top of a normal textbased editor.
These editors are sometimes referred to as structure-aware editors.

Despite having some clear advantages over traditional textbased editors, structural editors are not used among professional developers.
One reason for this is that the editors that are popular today have many years of engineering behind them.
There are no structural editors that come close in the amount of features or polish to that of IntelliJ, VSCode or Vim.
Structural editor still have unresolved issues in their design that needs to be addressed before they will see commercial use.
Deuce is a structure-aware editor developed as part of a research project at the University of Chicago ~\parencite{deuce}.
It suggests a hybrid approach that addresses some of the problems with current textbased work-flows.
I will be using this as a reference for what the modern structural editor might look like.

\chapter{The project}
During this thesis I will try to innovate on the current design of structural editors
and apply design principles used by editors like Vim and Emacs to develop an interface
that allows for efficient editing by users of alternative input methods.
A proof of concept prototype has already been developed, but I expect several iterations
to be needed before I am ready to work on the final prototype that will be used for testing.

\section{Methodology And Evaluation}
I foresee a few challenges when evaluating the results of the project.
The group of people that use systems like Talon is not very big, then it might be difficult define subjects
for user tests.
Therefore I plan to first show the correlation between usability for users of keyboards and users of voice control.
Once that has been established I can conduct user test and try to generalize that two voice control users.
This will hopefully not be too difficult, as most of the issues with current editing schemes are related to reliance on the mouse.

However testing on keyboard users also poses challenges.
When testing an application that is supposed to improve productivity, you can get quantitative data
by measuring the time of completion on a few representative tasks with both the new and the old solution, and comparing the results.
Editors like Vim have a notoriously high learning curve and it takes a long period of time using it before productivity increases significantly.
I suspected my solution will have a similar issue as it will likely be very different from both traditional editors
and the existing structural editors.
For user test I will have to rely on feedback provided by the users three there surveys or interviews.

One useful measurement for performance in this kind of application besides time to completion is number of actions required to perform a task.
For keyboard users an action is typically a keystroke, while for voice control users an action might be a single syllable.
To use this kind of measurements I will have to find a sensible way to weight certain actions differently from each other.
This involves answering questions like what is the cost of holding a modifying key compared to pressing a normal key.

Additionally when evaluating I will have to validate that all actions required to work in this environment are possible.

% \part{The project}

% \chapter{Planning the project}

% \part{Conclusion}

% \chapter{Results}

\backmatter{}
\printbibliography

\end{document}
