\documentclass[a4paper,english]{ifimaster}

\usepackage[utf8]{inputenc}
\usepackage{babel,duomasterforside}
\usepackage{hyperref}
\usepackage[backend=biber,style=authoryear]{biblatex}
\usepackage{minted}

% Style Chapters
\usepackage[T1]{fontenc}
\usepackage{titlesec, color}
\definecolor{gray75}{gray}{0.75}
\newcommand{\hsp}{\hspace{20pt}}
\titleformat{\chapter}[hang]{\Huge\bfseries}{\thechapter\hsp\textcolor{gray75}{|}\hsp}{0pt}{\Huge\bfseries}

\nonstopmode{}
\addbibresource{citations.bib}

\title{Master Thesis}
\subtitle{Leveraging Language Tooling for Better Voice Coding}
\author{Gaute Berge}

\begin{document}

\duoforside[dept={Department of Informatics},
program={Informatics: Programming and System Architecture},
long]

\frontmatter{}
\chapter*{Abstract}

\tableofcontents{}
\listoffigures{}
\listoftables{}

% \chapter*{Preface}

\mainmatter{}

\chapter{Introduction}

\section{Motivation}
\paragraph{Injuries}
Here I will cover research related to injuries such as RSI in the tech industry.
\paragraph{Learning Curve}
Here I explain breifly some of the difficulties people have when starting out with voice coding.
Learning curve is the biggers, and is extasterbated by the fact that you need
to be able to code in order to use the system profeciently, which leads to a 
chicken-egg situation.

\section{Research Questions}
\section{Approach}
\section{Chapter Overview}

\chapter{Background}
In this section I will cover the history of voice coding (Dragon, NatLink, Dragonfly, Caster etc\ldots)
as well as the state of the art language tooling. (LSP and TreeSitter).
I will also cover something about the Elm programming language, as I will be using that in my analysis.
% Donald Knuth often says smart stuff ~\parencite{Knuth:2007:CPA:1283920.1283929}.


\section{History of Voice Coding}

\paragraph{Dragon NaturallySpeaking}
The worlds first large-vocabulary general purpose dictation system.
Dragon is in many ways the foundation for many of tools like Talon.
Some voice systems use it as just a speach engine, but it also provides other functionality like mouse control.
Available for mac and windows, but discontinued for mac.% when??
The mac version is still available on eBay, and still works. % maybe something here about the download stuff.

\paragraph{NatLink}
While dragon alone is a very powerful dictation system, it's customization ability is limited.
Dragon does provide a simple scripting language in the pro edition, but it has very limited capabilities compared to
moden general purpose scripting languages.
In 1999, NatLink was developed to solve this problem. It is a macro system for dragon which lets the user write macros that can be triggered by voice commands
in python.~\parencite{gould2001implementation}
Later on, more high level APIs were built on top on natlink such as dragonfly and caster. % Not really precise. Need to look into history here.

\paragraph{Dragonfly}
Higher-level scripting api on top of natlink.

\paragraph{Caster}
Another layer on top of dragonfly. Not totally sure if worth discussing.

\paragraph{VoiceCode}
Seems like this was a popular dictation system at some point. There are papers on it, so might be worth discussing.

\section{Dictation}
\paragraph{Command First vs Data First}
Two different dication strategies.
Command first = Output only triggered when a command is triggered by speaking a particular phrase.
Data first = The user can speak without keywords. The system should infer capitalization and punctuation.
Here i will describe the tradeoff between these two.
\paragraph{Homophones}
Words that sound the same. Big challenge in speach recognition.


\section{TalonVoice}
Here I cover everything I can about how to use Talon and how the project is organized.
I wrote some in the essay, but this should be updated for the new API.

\subsection{Current state of the project}
\paragraph{Distribution and Monetization}
\paragraph{Community}
\paragraph{Learning Resources and Documentation}

\subsection{Using Talon}
In this section I will cover my workflow with Talon.
The level of detail I cover here should be reflected in when I cover~\nameref{wvc}
This is relevant because the reader needs to see the advantages and disadvantages of using talon compared to using a keboard.
\paragraph{Using The Terminal}
An essential part of the programming workflow. The terminal is accessible by default due to its text oriented interface.
\paragraph{Text Editing}
Here i will cover navigating a text editor with voice commands and eye-tracker.
\paragraph{Web Browsing}
Vimium and Surfingkeys. How are these plugins used.

\subsection{Writing Voice Commands with Talon}\label{wvc}
\paragraph{Rules}
\paragraph{Actions}
\paragraph{Contexts}
\paragraph{Dynamic Lists}

\subsection{Editor Integration}
Here I will cover how Talon can be integrated with Vim and IntelliJ, and also cover how to edit text without editor integration.
\paragraph{Shortcut Mappings}
The simplest integration. 
\paragraph{Server Client}
Send commands from talon over http (IntelliJ)
\paragraph{Vim}
Cover how vim is integrated in the community package.

\section{PL Theory}
As much of the work is related to analysing programs, I should cover some general stuff abuut languages.
Not sure how much detail is appropriate here, but could be a source for a lot of content.
Possible subjects: Parsing, type checking/inference, scopes, etc.

\section{Language Server Protocol}
Client-server based approach to reuable language IDE features.
I'll compare this to the traditional IDE approach (Eclipse / Intellij)
and discuss how this can be used by other tools than editors.

\section{TreeSitter}
TreeSitter is a system for incremental parsing developed by GitHub.
It was conceived as a solution to the problem of instant semantic syntax highlighting,
but is also used in the implementation of some language servers.

\section{Elm}
In this thisis I will be using Elm as the subject of the alysis.
Elm is a feature rich, but relatively simple language.
Some knowledge of Elm is useful for the reader, no experience is required.
Primarily I will cover the features of Elm that introduces new identifiers into the scope,
as these will be relevant when generating voice commands later.

\chapter{Methodology And Evaluation}
I might use a mix of quantitative and qualitative methods. (multi- strategy) % read creswell,2018
I could hold interviews with users of Talon to gather both types of data.

\paragraph{Speed Testing}
I can test the users programming speed with and without my system.
Here I must consider factors such as performance anxiety.
The programs they would be asked to dictate would have to be carefully crafted in order to have a consistent difficulty level
across the test. 
Should they dictate the same program twice, or two different? How can I be sure they are similarly difficult?
Should I change the order of the two tests?
This might well be a good measurement. I'll discuss why this paragraph.


\section{Qualitative}
I can gather qualitative data by having the user program using my system and ask them whether or not they find using it to be an improvement.
The disadvantage to this approach is that the system might have a learning curve which makes the initial impression
worse than it would have been over time.

\section{Quantitative}
One quantitative measure I could use is to analyse larger codebases to see how the predicted time to speak common identifiers change.
If I can count the number of syllables in a word I can compare the length of normal phrases need to produce a given function in a code base and compare
it to that of my system. This analysis can be weighted by the frequency of said identifiers.
The advantage of this approach is simplicity in that I don't depend on external users.
One interesting point here is to see how the result of this analysis relate to the results gathered from interviews.
\subsection{Data Collection}
Analyze the Elm implementation of real world app (https://github.com/rtfeldman/elm-spa-example)

\chapter{The project}

\chapter{Results}
\section{Interviews}
\section{Analysis}
\section{Comparing Results From Different Methods}


\backmatter{}
\printbibliography{}

\end{document}
